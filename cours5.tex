\documentclass{article}
\usepackage[utf8]{inputenc}
\usepackage[T1]{fontenc}
\usepackage[french]{babel}
\usepackage{minted}
\usepackage{amssymb}
\usepackage{amsmath}
\usepackage{geometry}
\usepackage{textcomp}
\usepackage{dsfont}
\makeatletter
\newsavebox\myboxA
\newsavebox\myboxB
\newlength\mylenA

\newcommand*\xoverline[2][0.75]{%
    \sbox{\myboxA}{$\m@th#2$}%
    \setbox\myboxB\null% Phantom box
    \ht\myboxB=\ht\myboxA%
    \dp\myboxB=\dp\myboxA%
    \wd\myboxB=#1\wd\myboxA% Scale phantom
    \sbox\myboxB{$\m@th\overline{\copy\myboxB}$}%  Overlined phantom
    \setlength\mylenA{\the\wd\myboxA}%   calc width diff
    \addtolength\mylenA{-\the\wd\myboxB}%
    \ifdim\wd\myboxB<\wd\myboxA%
       \rlap{\hskip 0.5\mylenA\usebox\myboxB}{\usebox\myboxA}%
    \else
        \hskip -0.5\mylenA\rlap{\usebox\myboxA}{\hskip 0.5\mylenA\usebox\myboxB}%
    \fi}
\makeatother
\newcommand{\textapprox}{\raisebox{0.5ex}{\texttildelow}}

  % added for comments and todo
\usepackage[colorinlistoftodos]{todonotes}


\geometry{total={210mm,297mm},
left=25mm,right=25mm,
top=25mm,bottom=25mm}

\title{%
  Probabilités \\
  \large Lois des grands nombres et TCL}

\date{29 mai 2018}

\begin{document}
\maketitle

\section{Loi des grands nombres}

\subsection{Énoncé}
Soient $X_1, X_2,...$ une suite de variables aléatoires indépendamment et
identiquement distribuées (\textit{iid}), telles que $\mu = \mathbb{E}(X_i)$.

Alors, on a :
\begin{align*}
  \lim_{n \to + \infty} \frac{1}{n} \sum_{i=1}^n X_i = \mu
\end{align*}

avec probabilité 1.

\subsection{Application}

Supposons $B_1,B_2,...$ des variables aléatoires de loi de Bernouli $b(p)$
indépendantes, alors :

$$ \frac{1}{n} \sum_{i=1}^n B_i \xrightarrow[n \to + \infty]{} p $$

\subsection{Exemple}
On répète indépendamment et dans les mêmes conditions une expérience.

On note $A_i$ : l'évènement $A$ s'est produit lors de l'expérience $i$.
Alors on a :

$$ \mathbb{P}(A) = \lim_{n \to + \infty} \frac{1}{n} \sum_{i=1}^n \mathds{1}_{A_i} $$

\section{Théorême central limite}

\subsection{Énoncé}
Soient $X_1, X_2,...$ une suite de variables aléatoires \textit{iid}, telles
que :
$ \mu = \mathbb{E}(X_i) $ et $\sigma^2 = \textit{Var}(X_i) > 0 $

On pose :
\begin{itemize}
  \item $S_n = \sum_{i=1}^n X_i$
  \item \[ \xoverline{X_n} = \frac{1}{n}S_n \]
  \item \[ Z_n = \frac{S_n - n \mu}{\sigma \sqrt{n}} = \frac{\xoverline{X_n} - \mu}{\sigma / \sqrt{n}}\]
\end{itemize}

Alors :

\begin{align*}
  \lim_{n \to + \infty} \mathbb{P}(Z_n \leq x) = \Phi (x)
\end{align*}

Où :

$$ \Phi (x) = \int_{- \infty}^x \frac{1}{\sqrt{2 \pi}} e^{-\frac{t^2}{2}} dt$$

On note le résultat $$Z_n \xrightarrow[n \to +\infty]{\mathcal{L}} \mathcal{N}(0,1)$$

\subsection{Remarque}
La vitesse de convergence de $\xoverline{X_n}$ vers $\mu$ est de l'ordre de $\frac{\sigma}{\sqrt{n}}$

\end{document}