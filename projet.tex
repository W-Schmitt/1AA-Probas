\documentclass{article}
\usepackage[utf8]{inputenc}
\usepackage[T1]{fontenc}
\usepackage[french]{babel}
\usepackage{minted}
\usepackage{graphicx}
\usepackage{amssymb}
\usepackage{amsmath}
\usepackage{pgfplots}
\usepackage{geometry}
\usepackage{textcomp}
\usepackage{dsfont}
\newcommand{\textapprox}{\raisebox{0.5ex}{\texttildelow}}

  % added for comments and todo
\usepackage[colorinlistoftodos]{todonotes}


\geometry{total={210mm,297mm},
left=25mm,right=25mm,
top=25mm,bottom=25mm}

\title{%
  Probabilités \\
  \large Informations sur le projet}
\date{23 mai 2018}

\begin{document}
\maketitle

\section{Arrivées}
\subsection{Recommandations}
\begin{itemize}
  \item Travailler en temps continu
  \item Développer avec des lois exponentielles
\end{itemize}

On suppose que les arrivées se font à des instants aléatoires $T_1,T_2,...,T_n$.

On pose $T_0=0$ et $X_k = T_k - T_{k-1}$

On peut prendre $X_k \textapprox \textit{Exp}(\lambda)$ et les $X_k$ indépendants entre eux.

$$ T_k = \sum_{i=1}^k X_i \textapprox \Gamma(k, \lambda)$$

Soit $N_t$ le nombre de requêtes reçues entre $[0,t]$.

\begin{align*}
  \mathbb{P}(N_t = k) &= \mathbb{P}(T_k \leq t < T_{k+1}) \\
  &= \mathbb{P}\Big( \frac{1}{t}T_k \leq 1 < \frac{1}{t}T_{k+1}\Big) \\
  & \frac{1}{t}T_k \textapprox \Gamma(k, \lambda t)  \\
  & \frac{1}{t}T_{k+1} \textapprox \Gamma(k+1, \lambda t) \\
  &= \frac{(\lambda t)^k}{k!}e^{-\lambda t}
\end{align*}

Cf feuille pour le graphique

\begin{align*}
  \mathbb{P}(\underbrace{N_7-N_5}_{\textapprox P(2 \lambda)} = 3) &= \frac{(2 \lambda)^3}{3!}e^{-2\lambda}
\end{align*}

\section{Service}
\paragraph{Enoncé :} le temps de service d'une requête est en moyenne de $\frac{1}{\mu}$

On suppose les temps de service indépendants entre eux et on les suppose de loi $\textit{Exp}(\mu)$.

\begin{itemize}
  \item $ S_0 = 0 $, $ S_1 \textapprox \textit{Exp}(\mu) $
  \item $ S_{i+1} = S_i + Y_i$ où $Y_i \textapprox \textit{Exp}(\mu)$, indépendant des autres $Y_j$.
\end{itemize}

$X_t$ le nombre de requêtes encore dans le système (en service + file d'attente) à l'instant t

\subsection{Un seul serveur}

Cf. feuille
\begin{itemize}
  \item 1 serveur
  \item Arrivées: $\textit{Exp}(\lambda)$
  \item Temps de service: $\textit{Exp}(\mu)$
\end{itemize}

Si on pose $\rho = \frac{\lambda}{\mu}$, en temps long :
$$ \mathbb{P}(X_t = k) = \frac{1-\rho}{1-\rho^{N+1}} \rho^k $$
avec $0 \leq k \leq N$.

Cette formule permet de vérifier les simulations.

Le nombre moyen de requêtes dans le système (en service + dans la file) est en temps long :
$$ \mathbb{E}(X_t) = \frac{\rho}{1-\rho^{N+1}} \Big( \frac{1-\rho^{N-1}}{1-\rho}- N\rho^N \Big)$$

\paragraph{Probabilité de perte}
$$ \mathbb{P}(X_t = N) = \frac{1- \rho}{1-\rho^{N+1}}\rho^N $$

\paragraph{Question} Si on sait qu'il y a eu $n$ arrivées de requêtes dans l'intervalle $[0,t]$,
comment se distribuent $T_1, T_2,...,T_n$ sur cet invervalle ?

Supposons $N_t = 1$, $s < t$

\begin{align*}
  \mathbb{P}(T_1 \leq s | N_t = 1) &= \frac{\mathbb{P}(T_1 \leq s, N_t = 1)}{\mathbb{P}(N_t = 1)} \\
  &= \frac{\mathbb{P}(T_1 \leq s, T_2 > t)}{\mathbb{P}(N_t = 1)} \\
  &= \frac{\mathbb{P}(T_1 \leq s, T_1 + X_2 > t)}{\mathbb{P}(N_t = 1)} \\
  &= \frac{\lambda s e^{- \lambda t}}{\lambda t e^{- \lambda t}} \\
  &= \frac{s}{t}
\end{align*}

Donc $(T_1 | N_t = 1) \textapprox \textit{Unif}(0,t)$

\end{document}